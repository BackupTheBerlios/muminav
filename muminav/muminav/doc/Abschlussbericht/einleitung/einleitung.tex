\chapter{Einleitung}

\section{Der Projektkontext} Mumie\ldots

\subsection{Dezentrale Systementwicklung am Beispiel GNU/Linux}

Das Projekt \glqq Mumie\grqq\ soll nach aktuellem Stand unter
einer Open Source-Lizenz entwickelt und ver�ffentlicht werden. Die
Zusammenarbeit zwischen den Universit�ten impliziert eine
dezentrale Entwicklung, an der auch das Projekt \glqq
MumieNav\grqq\ beteiligt ist. Das bei der Entwicklung entstehende
Produkt wird nicht zur \glqq Insell�sung\grqq\ , da es in ein
bestehendes Projekt integriert wird.\\
 Eine Aussicht auf die
Verwendung schon existierender Software besteht unter Umst�nden
bei der Realisierung der Datenschnittstelle. So gehen erste
�berlegungen in die Richtung, schon bestehende XML-Software zu
verwenden, um die vom Server �bertragenen Daten auszuwerten.

\subsection{Das Projekt Mumie}

Das Projekt wird vom Bundesministerium f�r Bildung und Forschung
gef�rdert. Es soll eine WWW-basierte, modulare Umgebung entwickelt
werden, welche sowohl den Lernenden als auch Dozenten eine
grafisch ansprechende und leicht zu bedienende Oberfl�che bietet.
So wird einerseits durch die Visualisierung mathematischer Inhalte
und dem interaktiven Umgang mit der Mathematik die Motivation
gef�rdert. Zus�tzlich werden einige lehr-- und lernbegleitende
Tools angeboten:

\begin{itemize}
\raggedright
\item Darstellung mathematischer Inhalte mit interaktiver
Multimedia-Unterst�tzung
\item Stoffnachbereitung, Wiederholungsunterst�tzung angeleitete und kommen\-tier\-te
�bungs\-auf\-gaben
\item Selbstkontrolle durch individuelle
Testumgebungen
\item Einf�hrung in mathematische Standardsoftwarepakete
\item individuelles Trainingscenter weiterf�hrender
Inhalte
\item Informationsplattform in der und vile und in die der es auch
noch kein wer was wie wo
\item Kommunikations-- und
Austauschangebote
\end{itemize}

Die urspr�ngliche Idee hierbei ist, mathematische Inhalte
darzustellen. Die Umgebung soll allerdings so gestaltet werden,
dass sie auch in anderen Disziplinen genutzt werden kann. Weitere
Schwerpunkte liegen auf der Individualisierung der Oberfl�che und
einer intelligenten Benutzerf�hrung, die verhindert, dass man in
der inhaltlichen F�lle des Gesamtangebotes die �bersicht verliert.

\section{Ziele} Unsere Aufgabe\ldots
