% Mit folgendem Header kann ich wahlweise 'latex bla.tex', oder 'pdflatex
% bla.tex' laufen lassen. Es wird dann eine dvi bzw. pdf Datei erzeugt. Die
% Bilder, die mit includegraphics importiert werden, werden ohne Dateiendung
% angegeben. Latex sucht sich dann entweder ein .eps bzw. .pdf/.png/.jpg File
% heraus...
%
%(http://www.techfak.uni-bielefeld.de/ags/ni/lectures/internstuff/howto/howto-la
%tex/howto-pdflatex.html)
\newif\ifpdf \ifx\pdfoutput\undefined
\pdffalse % we are not running pdflatex
\else
\pdfoutput=1 % we are running pdflatex
\pdfcompresslevel=9 % compression level for text and image;
\pdftrue \fi


\documentclass[12pt, headsepline]{scrreprt}
\usepackage[ngerman]{babel}

\ifpdf
  \usepackage[pdftex,xdvi]{graphicx}
  \usepackage[pdftex]{hyperref}     % option [colorlinks] erzeugt farbige Links statt

  \pdfinfo{
     /Title    (Abschlussbericht Muminav)
     /Author   (Michael Gl�ssel, Matthias Erche, J�rg K�ster)
     /Subject  ()
     /Keywords (tu-berlin, open source, java)
  }
  \usepackage[pdftex]{color} % farbige Umrandungen
\else
  \usepackage[dvips,xdvi]{graphicx}
  \usepackage[colorlinks]{hyperref}
  \usepackage{color}
\fi

\usepackage[automark,headsepline,plainheadsepline]{scrpage2}
\pagestyle{scrheadings}
%\ohead[\pagemark]{\pagemark}
\ohead[\headmark]{\headmark} \chead{\empty}



% ersetzt, s. o.
% \usepackage{color}            % um den Text auch farbig zu gestalten
% \usepackage[dvips]{graphicx}      % zum Einbinden von Grafiken, [<TREIBER>]

\usepackage{subfigure}                  % mehrere figures in einem floating object
\usepackage[latin1]{inputenc}       % wofuer?
\usepackage{textcomp}                   % fuer Sonderzeichen wie <Grad Celsius>
% \usepackage{hyperref}         % zum Einbinden von Querverweisen in pdf (als letztes
                    % auffuehren!)
%\graphicspath{{images/}}        % jetzt muss man nicht jedesmal den Pfad angeben




\title{Abschlussbericht \\ Muminav}

\author{
  Open-Source-Softwareprojekt\\
  Sommersemester 2002\\
  Technische Universit�t Berlin \\ \\
  \normalsize Michael Gl�ssel \\
  \normalsize Matthias Erche \\
  \normalsize J�rg K�ster\\
}

\date{\today}

\begin{document}     % here begins the actual document body
%\maketitle           % this actually creates the title block



\vspace*{\stretch{1}} \rule{\linewidth}{1mm}

\begin{flushright}
    \thispagestyle{empty}
    {\Large \textbf{Abschlu�bericht}\\}
 %   {\Huge \textbf{Muminav}\\}
\begin{figure}[htbp]
    \begin{flushright}
        \includegraphics[width=10cm]{figs/logo}
    \end{flushright}
\end{figure}
    \vspace{0.1cm}
    %\large



\begin{minipage}[b]{13cm}
\begin{raggedleft}
    Im Rahmen der Veranstaltung:\\
    Dezentrale Systementwicklung am Beispiel GNU/LINUX\\
    an der TU-Berlin --
    Sommersemester 2002\\
\end{raggedleft}
\end{minipage}
%\hspace*{21mm}
%\begin{minipage}[b]{\linewidth}
\begin{minipage}[t]{2cm}
\includegraphics[height=1.5cm]{figs/tulogo}
\end{minipage}





%    }

\end{flushright}


\rule{\linewidth}{1mm}





\vspace*{\stretch{2}}
\begin{center}
\large \textsc{Michael Gl\"assel, Matthias Erche, J�rg K�ster} \\
{\normalsize Betreuung: \textsc{Steffen Evers}}\\
\vspace{0.7cm} \today\\
%\begin{figure}[htbp]
%       \centering
%        \includegraphics[width=1.5cm]{figs/tulogo}
%\end{figure}
\end{center}


\begin{abstract}
\textbf{Zusammenfassung} blah \ldots
\end{abstract}

\tableofcontents


%--------------------------------------------------------------------

\chapter{Einleitung}

\section{Der Projektkontext} Mumie\ldots

\subsection{Dezentrale Systementwicklung am Beispiel GNU/Linux}

Das Projekt \glqq Mumie\grqq\ soll nach aktuellem Stand unter
einer Open Source-Lizenz entwickelt und ver�ffentlicht werden. Die
Zusammenarbeit zwischen den Universit�ten impliziert eine
dezentrale Entwicklung, an der auch das Projekt \glqq
MumieNav\grqq\ beteiligt ist. Das bei der Entwicklung entstehende
Produkt wird nicht zur \glqq Insell�sung\grqq\ , da es in ein
bestehendes Projekt integriert wird.\\
 Eine Aussicht auf die
Verwendung schon existierender Software besteht unter Umst�nden
bei der Realisierung der Datenschnittstelle. So gehen erste
�berlegungen in die Richtung, schon bestehende XML-Software zu
verwenden, um die vom Server �bertragenen Daten auszuwerten.

\subsection{Das Projekt Mumie}

Das Projekt wird vom Bundesministerium f�r Bildung und Forschung
gef�rdert. Es soll eine WWW-basierte, modulare Umgebung entwickelt
werden, welche sowohl den Lernenden als auch Dozenten eine
grafisch ansprechende und leicht zu bedienende Oberfl�che bietet.
So wird einerseits durch die Visualisierung mathematischer Inhalte
und dem interaktiven Umgang mit der Mathematik die Motivation
gef�rdert. Zus�tzlich werden einige lehr-- und lernbegleitende
Tools angeboten:

\begin{itemize}
\raggedright
\item Darstellung mathematischer Inhalte mit interaktiver
Multimedia-Unterst�tzung
\item Stoffnachbereitung, Wiederholungsunterst�tzung angeleitete und kommen\-tier\-te
�bungs\-auf\-gaben
\item Selbstkontrolle durch individuelle
Testumgebungen
\item Einf�hrung in mathematische Standardsoftwarepakete
\item individuelles Trainingscenter weiterf�hrender
Inhalte
\item Informationsplattform in der und vile und in die der es auch
noch kein wer was wie wo
\item Kommunikations-- und
Austauschangebote
\end{itemize}

Die urspr�ngliche Idee hierbei ist, mathematische Inhalte
darzustellen. Die Umgebung soll allerdings so gestaltet werden,
dass sie auch in anderen Disziplinen genutzt werden kann. Weitere
Schwerpunkte liegen auf der Individualisierung der Oberfl�che und
einer intelligenten Benutzerf�hrung, die verhindert, dass man in
der inhaltlichen F�lle des Gesamtangebotes die �bersicht verliert.

\section{Ziele} Unsere Aufgabe\ldots



%--------------------------------------------------------------------

\chapter{Organisation des Gesamtprojekts}

Regem��ige Meetings/Treffen

\section{Gruppenaufteilung}



\section{Vortr�ge}

\section{Workshops}

%--------------------------------------------------------------------

\chapter{Muminav}

\section{Lizenz}

Vorgabe war, die gesamte Entwicklungsarbeit als Open Source
Projekt durchzuf�hren. Dies schlie�t ein, die
Projektergebnisse unter einer Open Source Lizenz zu ver�ffentlichen.\\
Da wir unser Projekt in Zusammenarbeit mit der Mumie-Gruppe
entwickeln, mussten wir uns im Vorfeld mit Ihnen auf eine Open
Source Lizenz verst�ndigen, welche in das Gesamtprojekt Mumie
integrierbar ist.\\
Die GPL \cite{GPL} (GNU General Public License)



LGPL \cite{OpenSourceLicences},\cite{LGPL}.

\section{Entwicklungsumgebung}
\subsection{Projekthoster}
\subsection{Enwicklungswerkzeuge} Ant \cite{Ant2002}, JBuilder \cite{BerliOS}


%--------------------------------------------------------------------

\chapter{Beurteilung des Projekts}

\section{Diskussion des Status quo}

\section{Ausblick}

\bibliographystyle{bibstyle} % mit custom-bib erzeugter individueller Bibliographie-Style !!
\bibliography{bibliographie}


\end{document}
