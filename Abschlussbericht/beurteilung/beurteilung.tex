\chapter{Beurteilung des Projekts}

\section{Diskussion des Status quo}

\section{Einsatzf�higkeit}

\textsc{Muminav} wurde als hoch spezialisierte Softwarekomponente
f�r den Einsatz im Gesamtprojekt \textsc{Mumie} ins Leben gerufen.
Nichts desto trotz haben wir gro�en Wert darauf gelegt, die
Software so allgemein wie m�glich zu gestalten damit sie f�r
m�glichst viele andere Open-Source-Projekte interessant wird.
Letztendlich wurde, durch den Einsatz von Skins und XML als
Datenformat, aus der Aufgabe, ein Java-Applet zu programmieren,
welches Navigationsnetze f�r Mathematische Kurse im Internet
dynamisch erzeugt, ein Softwaremodul, welches f�r die Darstellung
jeglicher graphischer Darstellungen geeignet ist. Zum Beispiel
lie�en sich leicht Skins erstellen, mit denen man Flowcharts, E/R
Diagramme oder sogar UML darstellen kann.\\
Zus�tzlich ist die Software so ausgelegt, dass sie nicht auf
Applets beschr�nkt ist. Durch das Packaging in ein
Java-Swing-Panel l�sst es sich genau so gut in eine
Java-application einbauen.



\section{Ausblick}

F�r die Zukunft des Projekts haben wir uns vorgestellt, den
Abstraktionsgrad weiter zu steigern, damit es immer leichter wird,
seine eigenen Skins zu entwerfen.\\
Angedacht ist auch ein einfacher Skin-Editor, mit dem man sich
bequem per Maus seine eigenen Skins entwerfen kann.

Interessant w�re es auch, unsere Quelle, die wir bisher auf der
Java-Platform von \textsc{SUN} entwickelt und getestet haben, auf
die Kompatibilit�t zu Open-Source-Kompilern zu testen und
gegebenenfalls anzupassen. Damit k�nnten wir dann auch den
Bytecode unter einer entsprechenden Lizenz ver�ffentlichen.

Wir werden auch versuchen eine kleine Community ins Leben zu
rufen, welche gegenseitig Skins austauscht. Dabei k�nnte man die
\textsc{Muminav}-Homepage zu einem �ffentlich zug�nglichen
Skin-Archiv erweitern, in das jeder seine Skin einbringen und bei
Bedarf von der Kreativit�t anderer profitieren kann.

Was eigentlich gar nicht erw�hnt werden m�sste: \textsc{Muminav},
wie auch jedes andere Open-Source-Projekt, ist mit der
Fertigstellung der ersten Version nicht abgeschlossen. Vielmehr
wird es, solange es aus der Open-Source-Community Interesse und
Anregungen gibt, sich weiterentwickeln. Sei es durch entdeckte
Bugs oder Vorschl�ge f�r neue Features, die durch uns oder
Mitglieder aus der Open-Source-Community in das Projekt
einflie�en.
