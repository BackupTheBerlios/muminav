
\chapter{Organisation des Gesamtprojekts}

An \textsc{Muminav} haben im Rahmen der Veranstaltung
\emph{Dezentrale Systementwicklung am Beispiel GNU/LINUX}
mitgewirkt: J�rg K�ster, Matthias Erche und Michael Gl�ssel. Alle,
Studenten der TU-Berlin im Fach Informatik.

\section{Meilensteine}

Die Termin- und Sachzielverfolgung wurde mit Hilfe von 4
Meilensteinen sichergestellt.

\begin{itemize}
\item \textbf{8.\ Mai} Festlegung auf eine Projektaufgabe
\item \textbf{29.\ Mai} Bestandsaufnahme von relevanten Projekten
\item \textbf{19.\ Juni} Vorlage des L�sungskonzept
\item \textbf{1.\ Oktober} Zwischenbilanz und Prototyp
\end{itemize}



\section{Workshops}

Beim Workshop, der am 10.\ Juli stattfand, ging es um die
Nachbearbeitung, Analyse und Erg�nzung der gesammelten
Erfahrungen. Dazu wurden Gastredner eingeladen, die sich seit
l�ngerer Zeit in diesem Bereich bet�tigen und die mit ihrem Wissen
dazu beitragen k�nnen, die pers�nlichen Erfahrungen der Teilnehmer
in einem allgemeineren und objektiverem Zusammenhang zu sehen.


\section{Gruppenaufteilung}


Die Entwicklungs- und Implementierungsarbeit hat sich wie folgt
aufgeteilt:

\begin{itemize}
    \item \textbf{Matthias Erche} Entwerfen des Desings und der Inhalte der
    Projekt-Homepage, XML-Treeparser, Datenaustauschformat
    \item \textbf{J�rg K�ster} Zoomfunktionalit�t, Dokumentation
    und Skins
    \item \textbf{Michael Gl�ssel} Entwicklung des Grundger�sts,
    Mouseoververhalten und Tooltips
\end{itemize}

Zu jedem der ersten drei Meilensteine hat ein Gruppenmitglied
einen Vortrag gehalten.
