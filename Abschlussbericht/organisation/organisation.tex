
\chapter{Organisation des Gesamtprojekts}

An \textsc{Muminav} haben im Rahmen der Veranstaltung
\emph{Dezentrale Systementwicklung am Beispiel GNU/LINUX}
mitgewirkt: J�rg K�ster, Matthias Erche und Michael Gl�ssel, alles
Studenten der TU-Berlin im Fach Informatik.

\section{Meilensteine}

Die Termin- und Sachzielverfolgung wurde mit Hilfe von 4
Meilensteinen sichergestellt.

\begin{itemize}
\item \textbf{8.\ Mai} Festlegung auf eine Projektaufgabe
\item \textbf{29.\ Mai} Bestandsaufnahme von relevanten Projekten
\item \textbf{19.\ Juni} Vorlage des L�sungskonzept
\item \textbf{1.\ Oktober} Zwischenbilanz und Prototyp
\end{itemize}



\section{Workshops}

Beim Workshop, der am 10.\ Juli stattfand, ging es um die
Nachbearbeitung, Analyse und Erg�nzung der gesammelten
Erfahrungen. Dazu wurden Gastredner eingeladen, die sich seit
l�ngerer Zeit im Bereich Open Source (Softwareentwicklung) bet�tigen und die mit ihrem Wissen
dazu beitragen konnten, die pers�nlichen Erfahrungen der Teilnehmer
in einem allgemeineren und objektiverem Zusammenhang zu sehen.


\section{Gruppenaufteilung}


Die Entwicklungs- und Implementierungsarbeit hat sich wie folgt
aufgeteilt:

\begin{itemize}
    \item \textbf{Matthias Erche} Betreuung der Projektwebseite, Projekt Organisation, Entwurf des Datenaustauschformates, Programmierung der XML Verarbeitung, Dokumentation
    \item \textbf{J�rg K�ster} �berarbeitung der Zeichenengine, Zoomfunktionalit�t, Entwurf diverser (Zeichen-) Algorithmen, Skinprogrammierung, Dokumentation

    \item \textbf{Michael Gl�ssel} Umsetzung des Grundger�sts, Eventhandling, Projekt Organisation, Tooltips, Zeichenalgorithmen
\end{itemize}

Ausserdem haben sich alle drei bei der Konzeptentwicklung, der Recherche nach relevanten Projekten und dem Schreiben des Abschlussberichtes beteiligt. Zu jedem der ersten drei Meilensteine hat jeweils ein Gruppenmitglied einen Vortrag gehalten.
